\documentclass[a4paper,12pt,dvipdfm]{report}
\usepackage[brazil]{babel}
%\usepackage[latin1]{inputenc}
\usepackage[utf8]{inputenc}
\usepackage{indentfirst}
\usepackage[pdftex]{color,graphicx}
\usepackage{geometry}
\geometry{top=3cm ,bottom=2cm,left=2.5cm,right=2cm}
%\usepackage{dingbat}
\usepackage{amstext}
\usepackage{amscd}
\usepackage{amsfonts}
\usepackage{float}
\usepackage{textcomp}
\usepackage{amssymb}
%\usepackage{subfigure}
\usepackage{amsmath}
%\usepackage{amscd}
%\usepackage{graphics}
%\usepackage{picinpar}
\usepackage{multicol}
\usepackage{multirow}
%\usepackage{epigraph}
%\usepackage{natbib}
%\usepackage{setspace}
\usepackage{mathrsfs}
\usepackage{lscape}
%\usepackage{pdfpages}
\usepackage[normalem]{ulem}
%\usepackage{tikz}
%\usepackage[all]{xy}
\usepackage{enumerate}
\usepackage{mathdesign}
%\usepackage[T1]{fontenc}
%\usepackage{indentfirst}
%\usepackage[dvips]{color}
%\usepackage{caption}
%\usepackage{float}
%\usepackage[nottoc]{tocbibind} %inclui referencias no indice.
%\usepackage{enumerate}
%\usepackage{amsmath,amsfonts,amssymb}
%\usepackage{graphicx}
%\usepackage{verbatim}
\usepackage{amsthm}
%\usepackage{natbib}
%\usepackage{subfigure}
%\usepackage{setspace}







\renewcommand{\baselinestretch}{1.5}
\newcommand{\nulo}{\varnothing}
\newcommand{\x}{\times}
\newcommand{\carac}[1]{\mathcal{X}_{#1}}
%\newcommand{\ital}[1]{\textit{#1}}
%\newcommand{\negr}[1]{\textbf{#1}}
\newcommand{\duascolunas}[2]{\begin{minipage}{7cm} #1 \end{minipage}\hfill\begin{minipage}{7cm} #2 \end{minipage}\\\\} 
\newcommand {\expo}[1]{\exp{\left(#1\right)}}
\newcommand {\expi}[1]{\exp{i\left(#1\right)}}
\newcommand{\arc}[1]{\ensuremath{\overset{\frown}{\raisebox{0pt}[6pt]{#1}}}}
\newcommand*{\mes}{\ifthenelse{\the\month < 2}{Janeiro}
                  {\ifthenelse{\the\month < 3}{Fevereiro}
                  {\ifthenelse{\the\month < 4}{Março}
                  {\ifthenelse{\the\month < 5}{Abril}
                  {\ifthenelse{\the\month < 6}{Maio}
                  {\ifthenelse{\the\month < 7}{Junho}
                  {\ifthenelse{\the\month < 8}{Julho}
                  {\ifthenelse{\the\month < 9}{Agosto}
                  {\ifthenelse{\the\month < 10}{Setembro}
                  {\ifthenelse{\the\month < 11}{Outubro}
                  {\ifthenelse{\the\month < 12}{Novembro}{Dezembro}}}}}}}}}}}} %plota o mês atual
\newcommand {\sen}[1]{\sin{\left(#1\right)}}
\newcommand {\cossen}[1]{\cos{\left(#1\right)}}
\newcommand {\tg}[1]{\tan{\left(#1\right)}}
\newcommand {\cotg}[1]{\cot{\left(#1\right)}}
\newcommand {\seca}[1]{\sec{\left(#1\right)}}
\newcommand {\cossec}[1]{\csc{\left(#1\right)}}
\newcommand{\E}{\xi}
\newcommand {\refe}[1]{(\ref{#1})}
%\newcommand {\L}{\mathscr{L}}
\newcommand {\Ima}[1]{\mathrm{Im}{\left[#1\right]}}
\newcommand {\F}{\mathscr{F}}
%\newcommand {\L}{\mathscr{L}}
\newcommand {\om}{\Omega}
\newcommand {\fii}{\varphi}
\newcommand {\lap}{\Delta}
\newcommand {\gra}{\nabla}
\newcommand {\pc}{\vskip 1pc}
\newcommand {\fim}{\nl\rightline{$\square$}\vskip 2pc}
\newcommand {\nl}{\newline}
\newcommand {\cl}{\centerline}
\newcommand {\R}{\mathbb{R}}
\newcommand {\N}{\mathbb{N}}
\newcommand {\Z}{\mathbb{Z}}
\newcommand {\V}{\mathcal{V}^{hp}}
\newcommand {\Q}{\mathbb{Q}}
%\newcommand {\F}{\mathbb{F}}
\newcommand {\G}{\mathbb{G}}
\newcommand {\C}{\mathbb{C}}
\newcommand {\Ss}{\mathbb{S}}
\newcommand {\Ph}{\mathcal{P}_{\!h}}
\newcommand {\B}{\mathcal{B}}
\newcommand {\f}{\mathcal{F}}
\newcommand {\Lh}{\mathcal{L}}
\newcommand {\La}{\Lambda}
\newcommand{\Ri}{\Rightarrow}
\newcommand{\Li}{\Leftarrow}
\newcommand{\lr}{\Longleftrightarrow}
\newcommand{\dis}{\displaystyle}
\newcommand{\lon}{\longrightarrow}
\newcommand{\nin}{/\!\!\!\!\!\in}
%\newcommand {\la}{\lambda}
\newcommand {\al}{\alpha}
\newcommand {\bt}{\beta}
\newcommand {\til}{\widetilde}
\newcommand {\lb}{\linebreak}
\newcommand {\esp}{\hskip 1pc}
\newcommand {\be}{\nl\cl }
\newcommand {\normf}[3]{\Big| \!\! \; \Big|  \dfrac{#1}{#2} \Big| \!\! \; \Big|_{#3}}
\newcommand {\norma}[2] {{\parallel  \! #1 \!  \parallel}_{#2}}
\newcommand {\normp}[1] {{|\!|\!| #1 |\!|\!|}_{\! \Ph}}
\newcommand {\adsum}{\addcontentsline{toc}{subsection}}
\newcommand {\T}{\mathcal{T}}
\newcommand {\fecho}[1]{\overline{#1}}
\newcommand {\pref}[1]{(\ref{#1})}
\newcommand {\prcr}[2]{(#1\cup #2)_{\al,L}\rtimes\N}
\newcommand {\tp}[2]{\T(#1\cup #2)}
\newcommand{\mdc}{\text{mdc}}
\newcommand{\funcao}[5]{\begin{array}{cccc}
#1:&\!\!\!#2 & \rightarrow & #3 \\
  &\!\!\! #4 & \mapsto & #5
\end{array}}
\newcommand{\n}{{\bf n}}
\newcommand{\soma}[2]{\displaystyle\sum_{#1}^{#2}}
\newcommand {\flecha}[1] {\stackrel{#1 \rightarrow \infty}\longrightarrow}
\newcommand{\canto}[1]{\begin{flushright} #1 \end{flushright}}
\newcommand{\fd}{\vspace{-0,5cm} \begin{flushright} $\square$ \end{flushright} \vspace{-0,5cm}}
\newcommand{\der}{\partial}
%\newcommand{\sen}{{\rm  \ \! sen}}
\newcommand{\orb}[1]{\mathcal{O}\left(#1\right)}
\newcommand{\orbf}[1]{\mathcal{O}^{+}\left(#1\right)}
\newcommand{\orbp}[1]{\mathcal{O}^{-}\left(#1\right)}
\newcommand{\conjunto}[1]{\big\{#1\big\}}






\providecommand{\sin}{} \renewcommand{\sin}{\hspace{2pt}\textrm{sen\hspace{2pt}}}
\providecommand{\tan}{} \renewcommand{\tan}{\hspace{2pt}\textrm{tg\hspace{2pt}}}
\providecommand{\arctan}{} \renewcommand{\arctan}{\hspace{2pt}\textrm{arctg\hspace{2pt}}}
\providecommand{\arcsin}{} \renewcommand{\arcsin}{\hspace{2pt}\textrm{arcsen\hspace{2pt}}}







\theoremstyle{plain}
%\theoremstyle{definition}
\newtheorem{teorema}{Teorema}[chapter]
\newtheorem{corolario}[teorema]{Corol\'ario}
\newtheorem{lema}[teorema]{Lema}
\newtheorem{proposicao}[teorema]{Proposi\c{c}\~ao}
\newtheorem{definicao}[teorema]{Defini\c{c}\~{a}o}
\newtheorem{propriedade}[teorema]{Propriedades}
\newtheorem*{obs}{Observa\c{c}\~{a}o}
\newtheorem{ex}[teorema]{Exemplo}
\newtheorem*{solucao}{Solu\c{c}\~{a}o}
\newtheorem*{demo}{Demonstra\c{c}\~{a}o}






%\setcounter{secnumdepth}{5}
%\setcounter{tocdepth}{5}
\setlength{\parindent}{1.5cm}
%\onehalfspace
%\everymath{\displaystyle}
\setcounter{secnumdepth}{3}
%\voffset 3.8cm



\begin{document}
\DeclareGraphicsExtensions{.pdf,.png,.mps,.jpg}

\chapter{Exercícios}

\begin{definicao} Seja $f:M\to M$ um difeomorfismo parcialmente hiperbólico. Considerando $W^{ss(uu)}(x)\subseteq M$ a variedade estável (respectivamente instável) forte de $x$, chamamos de \textbf{disco estável} (respectivamente \textbf{instável}) \textbf{forte de $x$ de tamanho $k$}, a bola fechada $D_k^{ss(uu)}(x)\subseteq W^{ss(uu)}(x)$ de raio $k/2$, pela métrica em $W^{ss(uu)}(x)$, e centrada em $x$.
\end{definicao}



\begin{lema}\label{lemadeltadenso} Seja $f:M\to M$ um difeomorfismo parcialmente hiperbólico e $\varepsilon\in[0,1]$ dado, se $f$ for $ms-$minimal ou $mu-$minimal, então para todo $\delta>0$, existe um conjunto $W\subseteq\mathcal{X}(f)$ e $K>0$ suficientemente grande tal que $D_K^{ss(uu)}(x)$ é $\delta-$denso em $M$ para todo $x\in W$ e $m(W)>1-\varepsilon$.
\end{lema}

\begin{proof} Seja $\delta>0$, $x\in M$ um ponto qualquer e $D_{k}^{ss(uu)}(x)$ o disco estável (instável) forte de $x$ de tamanho $k$. Como $\overline{W^{ss(uu)}(x)}=M$, pois $f$ é $ss(uu)-$minimal, então existe $k_x\in\N$ tal que $D_{k}^{ss(uu)}(x)$ é $\delta-$denso em $M$. 

Pela continuidade das variedades estáveis (instáveis) forte, existe uma vizinhança $U_x$ de $x$ tal que para todo $y\in U_x$, existe $k_y\in\N$ tal que $D_{k}^{ss(uu)}(x)$ também é $\delta-$denso em $M$. Como $f$ é parcialmente hiperbólico e $x$ é um ponto qualquer de $M$, então $\cup_{x\in M}{U_x}$ é uma cobertura aberta de $M$, e pela compacidade de $M$ existe $n\in \N$ tal que $M\subseteq \cup_{i=1}^{n}{U_{x_i}}$. Seja $k_i\in \N$ tal que $D_{k_{x_i}}^{ss(uu)}(x_i)$ seja $\delta-$denso em $M$, tomemos $K=\max\conjunto{k_{x_1},k_{x_2},\cdots,k_{x_n}}$ e então para qualquer $x\in M$ temos que $x\in U_{x_i}$ para algum $i\in\N$, logo $D_{K}^{ss(uu)}(x)$ é $\delta-$denso em $M$.
\end{proof}


\begin{teorema} Seja $f:M\to M$ um difeomorfismo parcialmente hiperbólico preservando a medida de Lebesgue $m$. Se $f$ for $ms-$minimal ou $mu-$minimal, então $f$ é topologicamente \textit{mixing}.
\end{teorema}

\begin{proof} Vamos provar para o caso $mu-$minimal, e para o caso $ms-$minimal a demonstração é análoga.

Sejam $U,V\subseteq M$ dois abertos quaisquer. Tomemos $\varepsilon>0$ tal que $U$ contenha uma bola aberta $B$ de raio $\varepsilon$, e $D^{uu}_{\varepsilon}(x)\subseteq U$ para todo ponto $x\in B$. Como $B$ é aberto, então $b=m(B)>0$.

Seja $\delta>0$ tal que $V$ contenha uma bola de raio $\delta$. Como $f$ é $mu-$minimal, pelo Lema \ref{lemadeltadenso} existem $W\in M$ e $K>0$ suficientemente grande, tal que $D^{uu}_{K}(x)$ é $\delta-$denso para todo $x\in W$ e $m(W)>1-b$. Por hiperbolicidade, existe $n_0\in\N$ tal que $f^{n}\big(D^{uu}_{\varepsilon}(x)\big)\supseteq D^{uu}_{K}(x)$ para todo $n>n_0$ e para todo $x\in M$. Temos também que $m\big(f^{-n}(W)\big)=m(W)$ pois $f$ preserva a medida. Fixemos um $n\in \N$ onde $n>n_0$.

Afirmação: $f^{-n}(W)\cap B\neq\emptyset$. De fato, suponhamos que $f^{-n}(W)\cap B=\emptyset$. Daí, por aditividade da medida, temos que se $f^{-n}(W)\cap B=\emptyset$ então $m\big(f^{-n}(W)\cup B\big)=m\big(f^{-n}(W)\big)+m(B)>b+1-b=1$. Absurdo pois $f^{-n}(W)\cup B\subseteq M$ e $m(M)=1$. Logo $f^{-n}(W)\cap B\neq\emptyset$. 

Tomemos $z\in f^{-n}(W)\cap B$. Como $z\in f^{-n}(W)$ temos também que $f^n(z)\in W$, logo $f^{n}\big(D^{uu}_{\varepsilon}(z)\big)\supseteq D^{uu}_{K}\big(f^{n}(z)\big)$ é $\delta-$denso. Por escolha de $B$, temos $D^{uu}_{\varepsilon}(z)\subseteq U$ e por escolha de $\delta$, temos $f^{n}\big(D^{uu}_{\varepsilon}(z)\big)\cap V$ e como $n>n_0$ é um qualquer, então $f^n(U)\cup V\neq\emptyset$ para todo $n>n_0$. Como $U$ e $V$ foram tomados abertos quaisquer então $f$ é topologicamente \textbf{mixing}.
\end{proof}



\end{document}

\begin{definicao} Seja $f:M\to M$ um difeomorfismo parcialmente hiperbólico. Considerando $W^{ss(uu)}(x)\subseteq M$ a variedade estável forte (respectivamente instável forte) de $x$, chamamos de \textbf{disco estável forte} (respectivamente \textbf{instável forte}) \textbf{de $x$ de tamanho $K$}, a bola fechada $D_K^{ss(uu)}(x)\subseteq W^{ss(uu)}(x)$ de raio $K$, pela métrica em $W^{ss(uu)}(x)$, e centrada em $x$.
\end{definicao}

\begin{lema}\label{lemadeltadenso} Seja $f:M\to M$ um difeomorfismo parcialmente hiperbólico, se $f$ for $ss-$minimal ou $uu-$minimal, então dado $\delta>0$, existe um $K>0$ suficientemente grande tal que $D_K^{ss(uu)}(x)$ é $\delta-$denso em $M$ para todo $x\in M$.
\end{lema}

\begin{proof} Sejam $\delta>0$, $x\in M$ um ponto qualquer e $D_{k}^{ss(uu)}(x)$ o disco estável (instável) forte de $x$ de tamanho $k$. Como $\overline{W^{ss(uu)}(x)}=M$, pois $f$ é $ss(uu)-$minimal, então existe $k_x\in\N$ tal que $D_{k}^{ss(uu)}(x)$ é $\delta-$denso em $M$. 

Pela continuidade das variedades estáveis (instáveis) forte, existe uma vizinhança $U_x$ de $x$ tal que para todo $y\in U_x$, existe $k_y\in\N$ tal que $D_{k}^{ss(uu)}(x)$ também é $\delta-$denso em $M$. Como $f$ é parcialmente hiperbólico e $x$ é um ponto qualquer de $M$, então $\bigcup_{x\in M}{U_x}$ é uma cobertura aberta de $M$, e pela compacidade de $M$ existe $n\in \N$ tal que $M\subseteq \bigcup_{i=1}^{n}{U_{x_i}}$. Seja $k_i\in \N$ tal que $D_{k_{x_i}}^{ss(uu)}(x_i)$ é $\delta-$denso em $M$, tomemos $K=\max\conjunto{k_{x_1},k_{x_2},\cdots,k_{x_n}}$ e então para qualquer $x\in M$ temos que $x\in U_{x_i}$ para algum $i\in\N$, logo $D_{K}^{ss(uu)}(x)$ é $\delta-$denso em $M$.
\end{proof}


%\begin{figure}[H]
%\centering
%\includegraphics[height=5cm]{Imagem.png}
%\caption{legenda}
%\label{Imagem}
%\end{figure}

%\begin{equation*}

%\end{equation*}\vspace{0.1cm}

%\begin{eqnarray*}

%\end{eqnarray*}

%\begin{propriedade}

%\end{propriedade}

%\begin{definicao}

%\end{definicao}

%\begin{ex}

%\end{ex}

%\begin{solucao}

%\end{solucao}

%\begin{teorema}

%\end{teorema}

%\begin{proof}

%\end{proof}

%\begin{corolario}

%\end{corolario}

%\begin{proposicao}

%\end{proposicao}

%\begin{flushright}
%\begin{minipage}{7cm}
%\small
%\end{minipage}\vspace{1cm}
%\end{flushright}