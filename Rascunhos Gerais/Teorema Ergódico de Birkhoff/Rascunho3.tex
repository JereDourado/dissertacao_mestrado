\documentclass[a4paper,12pt,dvipdfm]{report}
\usepackage[brazil]{babel}
%\usepackage[latin1]{inputenc}
\usepackage[utf8]{inputenc}
\usepackage{indentfirst}
\usepackage[pdftex]{color,graphicx}
\usepackage{geometry}
\geometry{top=3cm ,bottom=2cm,left=2.5cm,right=2cm}
%\usepackage{dingbat}
\usepackage{amstext}
\usepackage{amscd}
\usepackage{amsfonts}
\usepackage{float}
\usepackage{textcomp}
\usepackage{amssymb}
%\usepackage{subfigure}
\usepackage{amsmath}
%\usepackage{amscd}
%\usepackage{graphics}
%\usepackage{picinpar}
\usepackage{multicol}
\usepackage{multirow}
%\usepackage{epigraph}
%\usepackage{natbib}
%\usepackage{setspace}
\usepackage{mathrsfs}
\usepackage{lscape}
%\usepackage{pdfpages}
\usepackage[normalem]{ulem}
%\usepackage{tikz}
%\usepackage[all]{xy}
\usepackage{enumerate}
\usepackage{mathdesign}
%\usepackage[T1]{fontenc}
%\usepackage{indentfirst}
%\usepackage[dvips]{color}
%\usepackage{caption}
%\usepackage{float}
%\usepackage[nottoc]{tocbibind} %inclui referencias no indice.
%\usepackage{enumerate}
%\usepackage{amsmath,amsfonts,amssymb}
%\usepackage{graphicx}
%\usepackage{verbatim}
\usepackage{amsthm}
%\usepackage{natbib}
%\usepackage{subfigure}
%\usepackage{setspace}







\renewcommand{\baselinestretch}{1.5}
\newcommand{\nulo}{\varnothing}
\newcommand{\x}{\times}
\newcommand{\carac}[1]{\mathcal{X}_{#1}}
%\newcommand{\ital}[1]{\textit{#1}}
%\newcommand{\negr}[1]{\textbf{#1}}
\newcommand{\duascolunas}[2]{\begin{minipage}{7cm} #1 \end{minipage}\hfill\begin{minipage}{7cm} #2 \end{minipage}\\\\} 
\newcommand {\expo}[1]{\exp{\left(#1\right)}}
\newcommand {\expi}[1]{\exp{i\left(#1\right)}}
\newcommand{\arc}[1]{\ensuremath{\overset{\frown}{\raisebox{0pt}[6pt]{#1}}}}
\newcommand*{\mes}{\ifthenelse{\the\month < 2}{Janeiro}
                  {\ifthenelse{\the\month < 3}{Fevereiro}
                  {\ifthenelse{\the\month < 4}{Março}
                  {\ifthenelse{\the\month < 5}{Abril}
                  {\ifthenelse{\the\month < 6}{Maio}
                  {\ifthenelse{\the\month < 7}{Junho}
                  {\ifthenelse{\the\month < 8}{Julho}
                  {\ifthenelse{\the\month < 9}{Agosto}
                  {\ifthenelse{\the\month < 10}{Setembro}
                  {\ifthenelse{\the\month < 11}{Outubro}
                  {\ifthenelse{\the\month < 12}{Novembro}{Dezembro}}}}}}}}}}}} %plota o mês atual
\newcommand {\sen}[1]{\sin{\left(#1\right)}}
\newcommand {\cossen}[1]{\cos{\left(#1\right)}}
\newcommand {\tg}[1]{\tan{\left(#1\right)}}
\newcommand {\cotg}[1]{\cot{\left(#1\right)}}
\newcommand {\seca}[1]{\sec{\left(#1\right)}}
\newcommand {\cossec}[1]{\csc{\left(#1\right)}}
\newcommand{\E}{\xi}
\newcommand {\refe}[1]{(\ref{#1})}
%\newcommand {\L}{\mathscr{L}}
\newcommand {\Ima}[1]{\mathrm{Im}{\left[#1\right]}}
\newcommand {\F}{\mathscr{F}}
%\newcommand {\L}{\mathscr{L}}
\newcommand {\om}{\Omega}
\newcommand {\fii}{\varphi}
\newcommand {\lap}{\Delta}
\newcommand {\gra}{\nabla}
\newcommand {\pc}{\vskip 1pc}
\newcommand {\fim}{\nl\rightline{$\square$}\vskip 2pc}
\newcommand {\nl}{\newline}
\newcommand {\cl}{\centerline}
\newcommand {\R}{\mathbb{R}}
\newcommand {\N}{\mathbb{N}}
\newcommand {\Z}{\mathbb{Z}}
\newcommand {\V}{\mathcal{V}^{hp}}
\newcommand {\Q}{\mathbb{Q}}
%\newcommand {\F}{\mathbb{F}}
\newcommand {\G}{\mathbb{G}}
\newcommand {\C}{\mathbb{C}}
\newcommand {\Ss}{\mathbb{S}}
\newcommand {\Ph}{\mathcal{P}_{\!h}}
\newcommand {\B}{\mathcal{B}}
\newcommand {\f}{\mathcal{F}}
\newcommand {\Lh}{\mathcal{L}}
\newcommand {\La}{\Lambda}
\newcommand{\Ri}{\Rightarrow}
\newcommand{\Li}{\Leftarrow}
\newcommand{\lr}{\Longleftrightarrow}
\newcommand{\dis}{\displaystyle}
\newcommand{\lon}{\longrightarrow}
\newcommand{\nin}{/\!\!\!\!\!\in}
%\newcommand {\la}{\lambda}
\newcommand {\al}{\alpha}
\newcommand {\bt}{\beta}
\newcommand {\til}{\widetilde}
\newcommand {\lb}{\linebreak}
\newcommand {\esp}{\hskip 1pc}
\newcommand {\be}{\nl\cl }
\newcommand {\normf}[3]{\Big| \!\! \; \Big|  \dfrac{#1}{#2} \Big| \!\! \; \Big|_{#3}}
\newcommand {\norma}[2] {{\parallel  \! #1 \!  \parallel}_{#2}}
\newcommand {\normp}[1] {{|\!|\!| #1 |\!|\!|}_{\! \Ph}}
\newcommand {\adsum}{\addcontentsline{toc}{subsection}}
\newcommand {\T}{\mathcal{T}}
\newcommand {\fecho}[1]{\overline{#1}}
\newcommand {\pref}[1]{(\ref{#1})}
\newcommand {\prcr}[2]{(#1\cup #2)_{\al,L}\rtimes\N}
\newcommand {\tp}[2]{\T(#1\cup #2)}
\newcommand{\mdc}{\text{mdc}}
\newcommand{\funcao}[5]{\begin{array}{cccc}
#1:&\!\!\!#2 & \rightarrow & #3 \\
  &\!\!\! #4 & \mapsto & #5
\end{array}}
\newcommand{\n}{{\bf n}}
\newcommand{\soma}[2]{\displaystyle\sum_{#1}^{#2}}
\newcommand {\flecha}[1] {\stackrel{#1 \rightarrow \infty}\longrightarrow}
\newcommand{\canto}[1]{\begin{flushright} #1 \end{flushright}}
\newcommand{\fd}{\vspace{-0,5cm} \begin{flushright} $\square$ \end{flushright} \vspace{-0,5cm}}
\newcommand{\der}{\partial}
%\newcommand{\sen}{{\rm  \ \! sen}}
\newcommand{\orb}[1]{\mathcal{O}\left(#1\right)}
\newcommand{\orbf}[1]{\mathcal{O}^{+}\left(#1\right)}
\newcommand{\orbp}[1]{\mathcal{O}^{-}\left(#1\right)}
\newcommand{\conjunto}[1]{\big\{#1\big\}}






\providecommand{\sin}{} \renewcommand{\sin}{\hspace{2pt}\textrm{sen\hspace{2pt}}}
\providecommand{\tan}{} \renewcommand{\tan}{\hspace{2pt}\textrm{tg\hspace{2pt}}}
\providecommand{\arctan}{} \renewcommand{\arctan}{\hspace{2pt}\textrm{arctg\hspace{2pt}}}
\providecommand{\arcsin}{} \renewcommand{\arcsin}{\hspace{2pt}\textrm{arcsen\hspace{2pt}}}







\theoremstyle{plain}
%\theoremstyle{definition}
\newtheorem{teorema}{Teorema}[chapter]
\newtheorem{corolario}[teorema]{Corol\'ario}
\newtheorem{lema}[teorema]{Lema}
\newtheorem{proposicao}[teorema]{Proposi\c{c}\~ao}
\newtheorem{definicao}[teorema]{Defini\c{c}\~{a}o}
\newtheorem{propriedade}[teorema]{Propriedades}
\newtheorem*{obs}{Observa\c{c}\~{a}o}
\newtheorem{ex}[teorema]{Exemplo}
\newtheorem*{solucao}{Solu\c{c}\~{a}o}
\newtheorem*{demo}{Demonstra\c{c}\~{a}o}






%\setcounter{secnumdepth}{5}
%\setcounter{tocdepth}{5}
\setlength{\parindent}{1.5cm}
%\onehalfspace
%\everymath{\displaystyle}
\setcounter{secnumdepth}{3}
%\voffset 3.8cm



\begin{document}
\DeclareGraphicsExtensions{.pdf,.png,.mps,.jpg}

\chapter{Teorema Ergódico de Birkhoff}

Seja $x\in M$ e um conjunto mensurável $E\subset M$, vamos tomar os $n$ primeiros iterados da orbita de $x$, e vamos considerar a fração desses iterados que estão em $E$:

\begin{eqnarray}
\tau_n(E,x) & = &\dfrac{1}{n}\#\big\{f^j(x)\in E:0\leq j\leq n-1\big\}\vspace{0.1cm}\nonumber\\
& = & \dfrac{1}{n}\sum_{j=0}^{n-1}\carac{E}(f^j(x))\label{L40}
\end{eqnarray}\vspace{0.1cm}

Onde $\carac{E}$ é a função característica do conjunto $E$, isto é, $\carac{E}(x)=1$ se $x\in E$ e $\carac{E}(x)=0$ se $x\notin E$.

Definimos o \textit{tempo médio de permanência} da orbita de $x$ em $E$, sendo o limite dessas frações fazendo $n$ tender ao infinito:

\begin{equation*}
\tau(E,x)=\lim_{n\to+\infty}\tau_n(E,x)
\end{equation*}\vspace{0.1cm}

Em geral, esse limite pode não existir.

\begin{lema}\label{tmpo}

Se o tempo médio de permanência existe para um ponto $x\in M$, então

\begin{equation*}
\tau(E,f(x))=\tau(E,x)
\end{equation*}\vspace{0.1cm}

\end{lema}

\begin{proof} De fato, por definição temos:
\begin{eqnarray*}
\tau(E,f(x)) & = & \lim_{n\to+\infty}\tau_n(E,f(x))\\
& = & \dfrac{1}{n}\lim_{n\to+\infty}\sum_{j=1}^{n}\carac{E}(f^j(x))\\
& = & \dfrac{1}{n}\lim_{n\to+\infty}\left(\sum_{j=0}^{n-1}\carac{E}(f^j(x))-\dfrac{1}{n}\big[\carac{E}(x)-\carac{E}(f^n(x))\big]\right)\\
& = & \dfrac{1}{n}\lim_{n\to+\infty}\sum_{j=0}^{n-1}\carac{E}(f^j(x))-\lim_{n\to\infty}\dfrac{1}{n}\big[\carac{E}(x)-\carac{E}(f^n(x))\big]\\
& = & \tau(E,x)-\lim_{n\to+\infty}\dfrac{1}{n}\big[\carac{E}(x)-\carac{E}(f^n(x))\big]
\end{eqnarray*}

Como a função característica é limitada, esse ultimo limite é igual a zero, e o lema está demonstrado.
\end{proof}

\begin{teorema}\label{teb1}

Seja $f:M\to M$ uma transformação mensurável e $\mu$ uma probabilidade invariante por $f$. Dado qualquer conjunto mensurável $E\subset M$, o tempo médio de permanência $\tau(E,x)$ existe em $\mu$-quase todo ponto $x\in M$. Além disso, 

\begin{equation*}
\int\tau(E,x)d\mu(x)=\mu(E).
\end{equation*}\vspace{0.1cm}

\end{teorema}

\begin{proof}

Seja $E\subset M$ um conjunto mensurável qualquer. Para cada $x\in M$, definamos:
\begin{eqnarray*}
\overline{\tau}(E,x) & = & \limsup_{n\to+\infty}\dfrac{1}{n}\#\big\{f^j(x)\in E:0\leq j\leq n-1\big\}\\
\underline{\tau}(E,x) & = & \liminf_{n\to+\infty}\dfrac{1}{n}\#\big\{f^j(x)\in E:0\leq j\leq n-1\big\}
\end{eqnarray*}\vspace{0.1cm}

Para todo $x\in M$ temos que

\begin{equation}\label{L42}
\overline{\tau}(E,f(x))=\overline{\tau}(E,x)\quad\text{e}\quad\underline{\tau}(E,f(x))=\underline{\tau}(E,x)
\end{equation}\vspace{0.1cm}

De fato, a demonstração é análoga a do lema \ref{tmpo}.

Para demonstrar a existência do tempo médio $\tau(E,x)$, basta mostrar que

\begin{equation}
\overline{\tau}(E,x)=\underline{\tau}(E,x)
\end{equation}\vspace{0.1cm}
para $\mu$-quase todo ponto $x\in M$.

Como $\overline{\tau}(E,x)\geq\underline{\tau}(E,x)$ para todo $x\in M$, então pra mostrar a igualdade, basta mostrar a seguinte desigualdade

\begin{equation}\label{L44}
\int\overline{\tau}(E,x)d\mu(x)\leq\mu(E)\leq\int\underline{\tau}(E,x)d\mu(x)
\end{equation}\vspace{0.1cm}

Vamos provar a primeira desigualdade em \eqref{L44}, e a segunda segue de argumento inteiramente análogo.

Dado $\varepsilon>0$ arbitrário. Por definição de $\limsup$, para cada $x\in M$ existem inteiros $t\geq 1$, tais que

\begin{equation}\label{L45}
\tau_t(E,x)\geq\overline{\tau}(E,x)-\varepsilon
\end{equation}\vspace{0.1cm}

Definamos $t:M \rightarrow \N$ uma função que leva o ponto $x\in M$ ao primeiro $t$ que satisfaça \eqref{L45}. Agora dividiremos a demonstração em dois casos.

\textbf{Caso Particular:} Suponhamos que a função $t$ seja limitada, ou seja, existe um $K\in\N$ tal que $t(x)\leq K$ para todo $x\in M$. Então fixemos um certo $n\in\N$. 

Dado $x \in M$, definamos uma sequencia $x_0,x_1,\cdots,x_s$ de pontos de $M$ e uma sequencia $t_0,t_1,\cdots,t_s$ de número naturais, do seguinte modo:

\begin{enumerate}
\item Tomemos $x_0=x$.
\item Depois fazemos $t_i=t(x_i)$ e $x_{i+1}=f^{t_i}(x_i)$.
\item Terminamos quando encontrarmos $x_s$ tal que $t_0+t_1+\cdots+t_s\geq n$.
\end{enumerate}

Note que aplicando $x_i=f^{t_0+t_1+\cdots+t_{i-1}}(x)$ em \eqref{L42}, temos $\overline{\tau}(E,x_i)=\overline{\tau}(E,x)$ para todo $i$. Pela definição de $\tau_t(E,x)$ em \eqref{L40}, temos

\begin{equation*}
\sum_{j=0}^{t_i-1}\carac{E}(f^j(x))=t_i\tau_{t_i}(E,x)
\end{equation*}\vspace{0.1cm}

Como essa equação vale para todo $x\in M$ em particular vale para todo $x_i$, e aplicando em \eqref{L45}, temos

\begin{eqnarray}
\sum_{j=0}^{t_i-1}\carac{E}(f^j(x_i)) & = &  t_i\tau_{t_i}(E,x_i)\nonumber\\
 & \geq & t_i(\overline{\tau}(E,x)-\varepsilon)\label{L46}
\end{eqnarray}\vspace{0.1cm}

Pela definição da sequencia $x_i$ temos que
\begin{eqnarray*}
&& x_0 = x\\
&& x_1 = f^{t_0}(x_0) = f^{t_0}(x)\\
&& x_2 = f^{t_1}(x_1) = f^{t_1}(f^{t_0}(x)) = f^{t_0+t_1}(x)\\
&& x_3 = f^{t_2}(x_2) = f^{t_2}(f^{t_0+t_1}(x)) = f^{t_0+t_1+t_2}(x)\\
&& \ \ \ \ \vdots\\
&& x_s = f^{t_0+t_1+\cdots+t_{s-1}}(x)
\end{eqnarray*}\vspace{0.1cm}
onde 
\begin{equation*}
t_0+t_1+\cdots+t_{s-1}<n-1
\end{equation*}\vspace{0.1cm}

E como pelo item 3 da definição da sequencia temos $t_0+t_1+\cdots+t_{s-1}\geq n-t_s \geq n-K$, podemos reescrever \eqref{L46}, colocando $x_i$ em função de $x$ para todo $i$, e somar todos os eles
\begin{eqnarray}
\sum_{j=0}^{n-1}\carac{E}(f^j(x)) & \geq & \left(t_0+t_1+\cdots+t_{s-1}\right)(\overline{\tau}(E,x)-\varepsilon)\nonumber\\
& \geq & (n-t_s)(\overline{\tau}(E,x)-\varepsilon)\nonumber\\
& \geq & (n-K)(\overline{\tau}(E,x)-\varepsilon)\label{L477}
\end{eqnarray}\vspace{0.1cm}

Essa desigualdade vale para todo $x\in M$, e como as funções características são integráveis, então
\begin{eqnarray*}
\int\sum_{j=0}^{n-1}\carac{E}(f^j(x))d\mu(x) & \geq & \int(n-K)(\overline{\tau}(E,x)-\varepsilon)d\mu(x)\\
\sum_{j=0}^{n-1}\int\carac{E}(f^j(x))d\mu(x) & \geq & (n-K)\int\overline{\tau}(E,x)d\mu(x)-(n-K)\varepsilon\int d\mu(x)\\
\sum_{j=0}^{n-1}\mu(E) & \geq & (n-K)\int\overline{\tau}(E,x)d\mu(x)-(n-K)\varepsilon\mu(M)\\
n\mu(E) & \geq & (n-K)\int\overline{\tau}(E,x)d\mu(x)-(n-K)\varepsilon\\
\mu(E) & \geq & \dfrac{(n-K)}{n}\int\overline{\tau}(E,x)d\mu(x)-\dfrac{(n-K)}{n}\varepsilon
\end{eqnarray*}\vspace{0.1cm}

Esse resultado vale para todo $n\in\N$, então passamos ao limite quando $n\to+\infty$
\begin{eqnarray*}
\mu(E) & \geq & \lim_{n\to+\infty}\left(\dfrac{(n-K)}{n}\int\overline{\tau}(E,x)d\mu(x)-\dfrac{(n-K)}{n}\varepsilon\right)\\
& = & \lim_{n\to+\infty}\left(\dfrac{(n-K)}{n}\right)\int\overline{\tau}(E,x)d\mu(x)-\lim_{n\to+\infty}\left(\dfrac{(n-K)}{n}\right)\varepsilon\\
& = & \int\overline{\tau}(E,x)d\mu(x)-\varepsilon
\end{eqnarray*}\vspace{0.1cm}

Como $\varepsilon>0$ é arbitrário, podemos  tendê-lo a 0, e temos

\begin{equation*}
\mu(E)\geq\int\overline{\tau}(E,x)d\mu(x)
\end{equation*}\vspace{0.1cm}

Terminando assim a demonstração para esse caso, onde a função $t$ é limitada.

\textbf{Caso Geral:} Dado $\varepsilon>0$ fixemos $K>>1$ suficientemente grande, de modo que

\begin{equation*}
\mu\left(\big\{y\in M; t(y)>K\big\}\right)<\varepsilon
\end{equation*}\vspace{0.1cm}	

Vamos mostrar que esse $K$ de fato existe. Definamos $A_n\subset M$ da seguinte maneira

\begin{equation*}
A_j=\big\{y\in M; t(y)\leq K\big\}
\end{equation*}\vspace{0.1cm}

É claro que $A_j\subset A_{j+1}$ e $\cup_{j\in\N}A_j=M$, então $\lim\mu(A_j)=\mu\left(\cup_{j\in\N}A_j\right)=\mu(M)=1$. Ou seja, para todo $\varepsilon>0$ dado, existe $j_0\in\N$, tal que $j>j_0$ implica $\mu(A_j)>1-\varepsilon$. Tomemos $K>j_0$, então $\mu(A_k)>1-\varepsilon$, isso implica $\mu(M-A_k)<\varepsilon$. Agora observe que $(M-A_k)=\big\{y\in M; t(y)\leq K\big\}$. Vamos chamar esse conjunto de $B$. 

De maneira similar ao cado particular, fixemos um certo $n\in\N$. Dado $x \in M$, definamos uma sequencia $x_0,x_1,\cdots,x_s$ de pontos de $M$ e uma sequencia $t_0,t_1,\cdots,t_s$ de número naturais, do seguinte modo:

\begin{enumerate}
\item Tomemos $x_0=x$.
\item Se $t(x_i)\leq K$, fazemos $t_i=t(x_i)$ e $x_{i+1}=f^{t_i}(x_i)$.
\item Se $t(x_i)> K$, fazemos $t_i=1$ e $x_{i+1}=f(x_i)$.
\item Terminamos quando encontrarmos $x_s$ tal que $t_0+t_1+\cdots+t_s\geq n$.
\end{enumerate}

Do caso particular, temos que para todo $i$, tal que $t(x_i)\leq K$, a desigualdade \eqref{L46} continua valendo

\begin{equation}\label{L47}
\sum_{j=0}^{t_i-1}\carac{E}(f^j(x_i))\geq t_i(\overline{\tau}(E,x)-\varepsilon)
\end{equation}\vspace{0.1cm}

A desigualdade acima, implica na seguinte

\begin{equation}\label{L48}
\sum_{j=0}^{t_i-1}\carac{E}(f^j(x_i))\geq t_i(\overline{\tau}(E,x)-\varepsilon)-\sum_{j=0}^{t_i-1}\carac{B}(f^j(x_i))
\end{equation}\vspace{0.1cm}

Essa desigualdade tem a vantagem de valer para todos os $x_i$. De fato , basta vermos que quando $x_i$ for tal que $t(x_i)\leq K$, o ultimo somatório fica igual a zero, e decorre diretamente de \eqref{L47}, e quando $t(x_i)> K$, temos que $t_{i}=1$ e esses somatórios terão apenas um elemento, ficando
\begin{eqnarray*}
\sum_{j=0}^{t_i-1}\carac{E}(f^j(x_i)) & \geq & t_i(\overline{\tau}(E,x)-\varepsilon)-\sum_{j=0}^{t_i-1}\carac{B}(f^j(x_i))\\
\sum_{j=0}^{0}\carac{E}(f^j(x_i)) & \geq & (\overline{\tau}(E,x)-\varepsilon)-\sum_{j=0}^{0}\carac{B}(f^j(x_i))\\\carac{E}(f(x_i)) & \geq & (\overline{\tau}(E,x)-\varepsilon)-\carac{B}(f(x_i))
\end{eqnarray*}\vspace{0.1cm}

Como $0\leq\carac{E}(f(x_i))\leq1$ por definição de função característica, $(\overline{\tau}(E,x)-\varepsilon)<1$ pois 
$\overline{\tau}(E,x)\leq1$ e $\varepsilon>0$, e por fim $\carac{B}(f(x_i))=1$ pela definição de $B$ e pela escolha do $x_i$, então podemos concluir que a desigualdade é verdadeira, pois o membro da esquerda é maior do que ou igual a zero, e o da direita é menor que zero.

Agora usando o mesmo método que usamos pra concluir \eqref{L477}, fazendo novamente $x_i = f^{t_0+t_1+\cdots+t_{i-1}}(x)$ para todo $i$, $x_s = f^{t_0+t_1+\cdots+t_{s-1}}(x)$ onde $t_0+t_1+\cdots+t_{s-1}<n-1$, e tendo em mente que pelo item 4 da definição da sequencia temos $t_0+t_1+\cdots+t_{s-1}\geq n-t_s \geq n-K$, e essa ultima desigualdade é verdade pois $t_i\leq K$ para todo $i$, então podemos generalizar

\begin{eqnarray}
\sum_{j=0}^{n-1}\carac{E}(f^j(x)) & \geq & \sum_{j=0}^{t_0+\cdots+t_{s-1}-1}\carac{E}(f^j(x))\nonumber\\
 & = & \sum_{j=0}^{t_0-1}\carac{E}(f^j(x_i))+\sum_{j=t_0}^{t_0+t_1-1}\carac{E}(f^j(x_i))+\sum_{j=t_0+t_1}^{t_0+t_1+t_2-1}\carac{E}(f^j(x_i))+\cdots\nonumber\\
 && \cdots+\sum_{t_0+\cdots+t_{s-2}}^{t_0+\cdots+t_{s-1}-1}\carac{E}(f^j(x_i))\nonumber\\	
 & \geq & \big[t_0(\overline{\tau}(E,x)-\varepsilon)-R(t_0)\big]+\big[t_1(\overline{\tau}(E,x)-\varepsilon)-R(t_2)\big]+\cdots\nonumber\\
 & & \cdots+\big[t_{s-1}(\overline{\tau}(E,x)-\varepsilon)-R(t_{s-1})\big]\label{L49}
% & \geq & \left(t_0+t_1+\cdots+t_{s-1}\right)(\overline{\tau}(E,x)-\varepsilon)\\
%& \geq & (n-t_s)(\overline{\tau}(E,x)-\varepsilon)\\
%& \geq & (n-K)(\overline{\tau}(E,x)-\varepsilon)
\end{eqnarray}\vspace{0.1cm}
onde $\displaystyle R(t_0)=\sum_{j=0}^{t_1-1}\carac{B}(f^j(x_i))\quad$ e $\quad\displaystyle R(t_i)=\sum_{j=t_0+\cdots+t_{i-1}}^{t_0+\cdots+t_i-1}\carac{B}(f^j(x_i))$, logo somando todos os $R(t_i)$ temos
\begin{eqnarray*}
\sum_{j=0}^{t_0+\cdots+t_{s-1}-1}R(t_j) & = & \sum_{j=0}^{t_0+\cdots+t_i-1}\carac{B}(f^j(x_i))\\
& \leq & \sum_{j=0}^{n-1}\carac{B}(f^j(x))
\end{eqnarray*}\vspace{0.1cm}

Podemos voltar para \eqref{L49}, e concluir

\begin{eqnarray*}
\sum_{j=0}^{n-1}\carac{E}(f^j(x)) & \geq & \big[t_0(\overline{\tau}(E,x)-\varepsilon)-R(t_0)\big]+\cdots+\big[t_{s-1}(\overline{\tau}(E,x)-\varepsilon)-R(t_{s-1})\big]\\
 & = & (t_0+\cdots+t_{s-1})(\overline{\tau}(E,x)-\varepsilon)-\sum_{j=0}^{t_0+\cdots+t_{s-1}-1}R(t_j)\\
 & \geq & (t_0+\cdots+t_{s-1})(\overline{\tau}(E,x)-\varepsilon)-\sum_{j=0}^{n-1}\carac{B}(f^j(x))\\
& \geq & (n-t_s)(\overline{\tau}(E,x)-\varepsilon)-\sum_{j=0}^{n-1}\carac{B}(f^j(x))\\
& \geq & (n-K)(\overline{\tau}(E,x)-\varepsilon)-\sum_{j=0}^{n-1}\carac{B}(f^j(x))
\end{eqnarray*}\vspace{0.1cm}

Essa desigualdade vale para todo $x\in M$, e como as funções características são integráveis, então
\begin{eqnarray*}
\int\sum_{j=0}^{n-1}\carac{E}(f^j(x))d\mu(x) & \geq & \int\left((n-K)(\overline{\tau}(E,x)-\varepsilon)-\sum_{j=0}^{n-1}\carac{B}(f^j(x))\right)d\mu(x)\\
\sum_{j=0}^{n-1}\int\carac{E}(f^j(x))d\mu(x) & \geq & (n-K)\int(\overline{\tau}(E,x)-\varepsilon)d\mu(x)-\sum_{j=0}^{n-1}\int\carac{B}(f^j(x))d\mu(x)\\
\sum_{j=0}^{n-1}\mu(E) & \geq & (n-K)\int\overline{\tau}(E,x)d\mu(x)-(n-K)\varepsilon\mu(M)-\sum_{j=0}^{n-1}\mu(B)\\
n\mu(E) & \geq & (n-K)\int\overline{\tau}(E,x)d\mu(x)-(n-K)\varepsilon-n\mu(B)\\
\mu(E) & \geq & \dfrac{(n-K)}{n}\int\overline{\tau}(E,x)d\mu(x)-\dfrac{(n-K)}{n}\varepsilon-\mu(B)
\end{eqnarray*}\vspace{0.1cm}

Esse resultado vale para todo $n\in\N$, então passamos ao limite quando $n\to+\infty$, e como $\mu(B)<3$, temos
\begin{eqnarray*}
\mu(E) & \geq & \lim_{n\to+\infty}\left(\dfrac{(n-K)}{n}\int\overline{\tau}(E,x)d\mu(x)-\dfrac{(n-K)}{n}\varepsilon-\varepsilon\right)\\
& = & \lim_{n\to+\infty}\left(\dfrac{(n-K)}{n}\right)\int\overline{\tau}(E,x)d\mu(x)-\lim_{n\to+\infty}\left(\dfrac{(n-K)}{n}\right)\varepsilon-\lim_{n\to+\infty}\varepsilon\\
& = & \int\overline{\tau}(E,x)d\mu(x)-2\varepsilon
\end{eqnarray*}\vspace{0.1cm}

Como $\varepsilon>0$ é arbitrário, podemos  tendê-lo a 0, e temos

\begin{equation*}
\mu(E)\geq\int\overline{\tau}(E,x)d\mu(x)
\end{equation*}\vspace{0.1cm}

Isso completa a demonstração do caso geral do teorema.

\end{proof}

\begin{teorema}\label{teb}

{\bf (Teorema Ergódico de Birkhoff)} Seja $f : M \to M$ uma transformação mensurável e $\mu$ uma probabilidade invariante por $f$. Dada qualquer função integrável $\varphi : M \to \R$ o limite 

\begin{equation*}
\tilde{\varphi}(x) = \lim_{n\to+\infty}\dfrac{1}{n}\sum_{j=0}^{n-1}{\varphi(f^j(x))}
\end{equation*}

existe em $\mu$-quase todo ponto $x\in M$. Além disso,

\begin{equation*}
\int{\tilde{\varphi}(x)d\mu(x)}=\int{\varphi(x)d\mu(x)}.
\end{equation*}

\end{teorema}

\begin{proof}
Este enunciado mais geral pode ser provado usando uma versão um pouco mais elaborada do argumento usado pra provar \ref{teb1}, que não apresentaremos aqui.
\end{proof}

\begin{obs}
O Teorema \ref{teb1} é o caso particular do Teorema Ergódico de Birkhoff \eqref{teb} quando $\varphi=\carac{E}$, a função característica do conjunto $E$.
\end{obs}

\end{document}



%\begin{figure}[H]
%\centering
%\includegraphics[height=5cm]{Imagem.png}
%\caption{legenda}
%\label{Imagem}
%\end{figure}

%\begin{equation*}

%\end{equation*}\vspace{0.1cm}

%\begin{eqnarray*}

%\end{eqnarray*}\vspace{0.1cm}

%\begin{propriedade}

%\end{propriedade}

%\begin{definicao}

%\end{definicao}

%\begin{ex}

%\end{ex}

%\begin{solucao}

%\end{solucao}

%\begin{teorema}

%\end{teorema}

%\begin{proof}

%\end{proof}

%\begin{corolario}

%\end{corolario}

%\begin{flushright}
%\begin{minipage}{7cm}
%\small
%\end{minipage}\vspace{1cm}
%\end{flushright}