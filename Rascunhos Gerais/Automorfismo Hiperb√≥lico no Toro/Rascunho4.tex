\documentclass[12pt,a4paper,oneside]{report}%
\usepackage{amssymb}
\usepackage{amsmath,accents}
\usepackage{amsfonts}
\usepackage[brazil]{babel}
\usepackage{graphicx}
%\usepackage[latin1]{inputenc}
\usepackage{latexsym}
\usepackage{wrapfig}
\usepackage{makeidx}
\setlength{\topmargin}{-2cm}
\setlength{\oddsidemargin}{0cm}
\setlength{\evensidemargin}{0cm}
\setlength{\textwidth}{17cm}
\setlength{\textheight}{25.7cm}
\flushbottom


%%%%%%%%%%%%%%%%%%%%%%%%%%%%%%%%%%%%%%%%%%%%%%%
%%%%%%%%%%%%%COMEÇO DO MEU PREAMBULO%%%%%%%%%%%
%%%%%%%%%%%%%%%%%%%%%%%%%%%%%%%%%%%%%%%%%%%%%%%

\usepackage[utf8]{inputenc}
\usepackage{indentfirst}
\usepackage{indentfirst}
%\usepackage[pdftex]{color,graphicx}
\usepackage{amstext}
\usepackage{amscd}
\usepackage{float}
\usepackage{textcomp}
\usepackage{multicol}
\usepackage{multirow}
\usepackage{mathrsfs}
\usepackage{lscape}
\usepackage[normalem]{ulem}
\usepackage{enumerate}
\usepackage{mathdesign}
\usepackage{amsthm}
\usepackage[all]{xy}
\usepackage{accents}
\usepackage[hyphens]{url}
\usepackage{hyperref}
\usepackage{pdfpages}





\renewcommand{\baselinestretch}{1.5}
\newcommand{\nulo}{\varnothing}
\newcommand{\x}{\times}
\newcommand{\carac}[1]{\mathcal{X}_{#1}}
%\newcommand{\ital}[1]{\textit{#1}}
%\newcommand{\negr}[1]{\textbf{#1}}
\newcommand{\duascolunas}[2]{\begin{minipage}{7cm} #1 \end{minipage}\hfill\begin{minipage}{7cm} #2 \end{minipage}\\\\} 
\newcommand {\expo}[1]{\exp{\left(#1\right)}}
\newcommand {\expi}[1]{\exp{i\left(#1\right)}}
\newcommand{\arc}[1]{\ensuremath{\overset{\frown}{\raisebox{0pt}[6pt]{#1}}}}
\newcommand*{\mes}{\ifthenelse{\the\month < 2}{Janeiro}
                  {\ifthenelse{\the\month < 3}{Fevereiro}
                  {\ifthenelse{\the\month < 4}{Março}
                  {\ifthenelse{\the\month < 5}{Abril}
                  {\ifthenelse{\the\month < 6}{Maio}
                  {\ifthenelse{\the\month < 7}{Junho}
                  {\ifthenelse{\the\month < 8}{Julho}
                  {\ifthenelse{\the\month < 9}{Agosto}
                  {\ifthenelse{\the\month < 10}{Setembro}
                  {\ifthenelse{\the\month < 11}{Outubro}
                  {\ifthenelse{\the\month < 12}{Novembro}{Dezembro}}}}}}}}}}}} %plota o mês atual
\newcommand {\sen}[1]{\sin{\left(#1\right)}}
\newcommand {\cossen}[1]{\cos{\left(#1\right)}}
\newcommand {\tg}[1]{\tan{\left(#1\right)}}
\newcommand {\cotg}[1]{\cot{\left(#1\right)}}
\newcommand {\seca}[1]{\sec{\left(#1\right)}}
\newcommand {\cossec}[1]{\csc{\left(#1\right)}}
\newcommand{\E}{\xi}
%\newcommand {\L}{\mathscr{L}}
\newcommand {\Ima}[1]{\mathrm{Im}{\left[#1\right]}}
\newcommand {\F}{\mathscr{F}}
%\newcommand {\L}{\mathscr{L}}
\newcommand {\om}{\Omega}
\newcommand {\fii}{\varphi}
\newcommand {\lap}{\Delta}
\newcommand {\gra}{\nabla}
\newcommand {\pc}{\vskip 1pc}
\newcommand {\fim}{\nl\rightline{$\square$}\vskip 2pc}
\newcommand {\nl}{\newline}
\newcommand {\cl}{\centerline}
\newcommand {\R}{\mathbb{R}}
\newcommand {\N}{\mathbb{N}}
\newcommand {\Z}{\mathbb{Z}}
\newcommand {\V}{\mathcal{V}^{hp}}
\newcommand {\Q}{\mathbb{Q}}
%\newcommand {\F}{\mathbb{F}}
\newcommand {\G}{\mathbb{G}}
\newcommand {\C}{\mathbb{C}}
\newcommand {\Ss}{\mathbb{S}}
\newcommand {\Ph}{\mathcal{P}_{\!h}}
\newcommand {\B}{\mathcal{B}}
\newcommand {\f}{\mathcal{F}}
\newcommand {\Lh}{\mathcal{L}}
\newcommand {\La}{\Lambda}
\newcommand{\Ri}{\Rightarrow}
\newcommand{\Li}{\Leftarrow}
\newcommand{\lr}{\Longleftrightarrow}
\newcommand{\dis}{\displaystyle}
\newcommand{\lon}{\longrightarrow}
\newcommand{\nin}{/\!\!\!\!\!\in}
%\newcommand {\la}{\lambda}
\newcommand {\al}{\alpha}
\newcommand {\bt}{\beta}
\newcommand {\til}{\widetilde}
\newcommand {\lb}{\linebreak}
\newcommand {\esp}{\hskip 1pc}
\newcommand {\be}{\nl\cl }
\newcommand {\normf}[3]{\Big| \!\! \; \Big|  \dfrac{#1}{#2} \Big| \!\! \; \Big|_{#3}}
\newcommand {\norma}[2] {{\parallel  \! #1 \!  \parallel}_{#2}}
\newcommand {\normp}[1] {{|\!|\!| #1 |\!|\!|}_{\! \Ph}}
\newcommand {\adsum}{\addcontentsline{toc}{subsection}}
\newcommand {\T}{\mathbb{T}}
\newcommand {\fecho}[1]{\overline{#1}}
%\newcommand {\interior}[1]{\accentset{\circ}{#1}}
\newcommand {\interior}[1]{\accentset{\smash{\raisebox{-0.12ex}{$\scriptstyle\circ$}}}{#1}\rule{0pt}{2.3ex}}
\newcommand {\pref}[1]{(\ref{#1})}
\newcommand {\prcr}[2]{(#1\cup #2)_{\al,L}\rtimes\N}
\newcommand {\tp}[2]{\T(#1\cup #2)}
\newcommand{\mdc}{\text{mdc}}
\newcommand{\funcao}[5]{\begin{array}{cccc}
#1:&\!\!\!#2 & \rightarrow & #3 \\
  &\!\!\! #4 & \mapsto & #5
\end{array}}
\newcommand{\n}{{\bf n}}
\newcommand{\soma}[2]{\displaystyle\sum_{#1}^{#2}}
\newcommand {\flecha}[1] {\stackrel{#1 \rightarrow \infty}\longrightarrow}
\newcommand{\canto}[1]{\begin{flushright} #1 \end{flushright}}
\newcommand{\fd}{\vspace{-0,5cm} \begin{flushright} $\square$ \end{flushright} \vspace{-0,5cm}}
\newcommand{\der}{\partial}
%\newcommand{\sen}{{\rm  \ \! sen}}
\newcommand{\orb}[1]{\mathcal{O}\left(#1\right)}
\newcommand{\orbf}[1]{\mathcal{O}^{+}\left(#1\right)}
\newcommand{\orbp}[1]{\mathcal{O}^{-}\left(#1\right)}
\newcommand{\conjunto}[1]{\big\{#1\big\}}
\newcommand{\rec}[1]{\mathcal{R}\left(#1\right)}
\newcommand{\recc}[1]{\mathcal{RC}\left(#1\right)}
\newcommand{\prob}[1]{\mathcal{M}_1\left(#1\right)}
\newcommand{\diff}{\operatorname{Diff}}



\providecommand{\sin}{} \renewcommand{\sin}{\hspace{2pt}\textrm{sen\hspace{2pt}}}
\providecommand{\tan}{} \renewcommand{\tan}{\hspace{2pt}\textrm{tg\hspace{2pt}}}
\providecommand{\arctan}{} \renewcommand{\arctan}{\hspace{2pt}\textrm{arctg\hspace{2pt}}}
\providecommand{\arcsin}{} \renewcommand{\arcsin}{\hspace{2pt}\textrm{arcsen\hspace{2pt}}}


%%%%%%%%%%%%%%%%%%%%%%%%%%%%%%%%%%%%%%%%%%%%%%%
%%%%%%%%%%%%%FIM DO MEU PREAMBULO%%%%%%%%%%%%%%
%%%%%%%%%%%%%%%%%%%%%%%%%%%%%%%%%%%%%%%%%%%%%%%

\newtheorem{teorema}{Teorema}[chapter]
\newtheorem{lema}[teorema]{Lema}
\newtheorem{proposicao}[teorema]{Proposi\c{c}\~ao}
\newtheorem{corolario}[teorema]{Corol\'ario}
\newtheorem{definicao}[teorema]{Defini\c c\~{a}o}
\newtheorem{exercicio}[teorema]{Exerc\'icio}
\newtheorem{ex}{Exemplo}[chapter]
\newtheorem*{solucao}{Solu\c{c}\~{a}o}
\newtheorem{obs}[teorema]{Observa\c{c}\~{a}o}


%\newtheorem{teorema}{Teorema}[chapter]
%\newtheorem{lema}{Lema}[chapter]
%\newtheorem{proposicao}{Proposi\c{c}\~ao}[chapter]
%\newtheorem{corolario}{Corol\'ario}[chapter]
%\newtheorem{definicao}{Defini\c c\~{a}o}[chapter]
%\newtheorem{exercicio}{Exerc\'icio}[chapter]
%\newtheorem{ex}{Exemplo}[chapter]
%\newtheorem*{solucao}{Solu\c{c}\~{a}o}
%\newtheorem{obs}{Observa\c{c}\~{a}o}[chapter]


\makeindex
\pagestyle{myheadings}



\begin{document}

\DeclareGraphicsExtensions{.jpg,.pdf,.eps,.png} \pagenumbering{roman} 
\pagestyle{plain}

\chapter{Automorfismo Hiperbólico no Toro}

\begin{definicao} Um sistema dinâmico $f:M\to M$ é chamado \textbf{minimal} se $\fecho{\orb{x}}=M$, para todo $x\in M$, ou seja, a orbita de todo ponto $x \in M$ é densa em $M$.

\end{definicao}

\section{Rotação Irracional no Circulo}

Seja $S^1=\{z\in\C;|z|=1\}=\{e^{2\pi it};t\in\R\}$ a esfera unitária em $\C$ e o conjunto $\R/\Z=\{[x]\in [0,1);x\thicksim x'\Leftrightarrow x-x'\in\Z\}$, onde $[x]$ é a classe de equivalência pela relação $\thicksim$. Para facilitar a notação, ao invés de escrevermos $[x]\in\R/\Z$, escreveremos apenas $x\in\R/\Z$ onde o $x$ estará representando sua classe de equivalência $x\ (\mod 1)$.

\begin{proposicao} O grupo multiplicativo $S^1$ é isomorfo ao grupo aditivo $\R/\Z$.

\end{proposicao}

\begin{proof} Definamos a seguinte função:
\begin{equation*}
\funcao{h}{\R/\Z}{S^1}{t}{e^{2\pi it}}
\end{equation*}

$i)$ $h$ é um homomorfismo: De fato, seja $x,y\in \R/\Z$, então
\begin{eqnarray*}
h(x+y) & = & e^{2\pi i (x+y)}\\
 & = & e^{2\pi ix + 2\pi iy}\\
 & = & e^{2\pi ix}e^{2\pi iy}\\
 & = & h(x)h(y)
\end{eqnarray*}

$ii)$ $h$ é injetora: De fato, seja $x,y\in \R/\Z$ tal que $h(x)=h(y)$, então $e^{2\pi ix}=e^{2\pi iy}$\newline $\Rightarrow$ $e^{2\pi i(x-y)}=1$ $\Rightarrow$ $2\pi i(x-y)=\ln(1)$ $\Rightarrow$ $2\pi i(x-y)=0$ $\Rightarrow$ $x-y=0$ $\Rightarrow$ $x=y$.

$iii)$ $h$ é sobrejetora: De fato, seja $y\in S^1$, então $y=e^{2\pi ix}$ para algum $x\in\R$, logo existe $x\in\R/\Z$ tal que $h(x)=e^{2\pi ix}=y$.

Portanto, $h$ é um isomorfismo de grupos, ou seja, $\R/\Z$ é isomorfo a $S^1$. 
\end{proof}

Seja $\alpha\in\R$ e $R_{\alpha}:S^1\to S^1$ uma dinâmica em $S^1$, tal que $R_{\alpha}(z)=e^{2\pi i\alpha}z$, como todo elemento $z\in S^1$ é da forma $z=e^{2\pi it}$ para algum $t\in \R$, então $R_{\alpha}(z)=e^{2\pi i\alpha}z=e^{2\pi i\alpha}e^{2\pi it}=e^{2\pi i(t+\alpha)}$ rotação pelo angulo $2\pi\alpha$. Chamaremos essa função de \textbf{rotação pelo angulo $\alpha$}. Agora seja $T_{\alpha}:\R/\Z\to \R/\Z$ uma dinâmica em $\R/\Z$, tal que $T_{\alpha}(x)=x+\alpha\ (\mod 1)$.

\begin{proposicao} A função $h:\R/\Z\to S^1$ tal que $h(t)=e^{2\pi it}$ é uma conjugação de $T_{\alpha}$ e $R_{\alpha}$

\end{proposicao}

\begin{proof} Vamos mostrar que $h\circ T_{\alpha}=R_{\alpha}\circ h$.
\begin{eqnarray*}
(h\circ T_{\alpha})(x) & = & h(T_{\alpha}(x))\\
 & = & h(x+\alpha)\\
 & = & e^{2\pi i(x+\alpha)}\\
 & = & e^{2\pi ix}e^{2\pi i\alpha}\\
 & = & R_{\alpha}(e^{2\pi ix})\\
 & = & R_{\alpha}(h(x))\\
 & = & (R_{\alpha}\circ h)(x)
\end{eqnarray*}
\end{proof}

\begin{proposicao} Se $\alpha$ for um número racional, então $R_{\alpha}$ é periódica para todo 

\end{proposicao}

\begin{proof} Vamos mostrar que $h\circ T_{\alpha}=R_{\alpha}\circ h$.
\begin{eqnarray*}
(h\circ T_{\alpha})(x) & = & h(T_{\alpha}(x))\\
 & = & h(x+\alpha)\\
 & = & e^{2\pi i(x+\alpha)}\\
 & = & e^{2\pi ix}e^{2\pi i\alpha}\\
 & = & R_{\alpha}(e^{2\pi ix})\\
 & = & R_{\alpha}(h(x))\\
 & = & (R_{\alpha}\circ h)(x)
\end{eqnarray*}
\end{proof}












\end{document}



%\begin{figure}[H]
%\centering
%\includegraphics[height=5cm]{Imagem.png}
%\caption{legenda}
%\label{Imagem}
%\end{figure}

%\begin{equation*}

%\end{equation*}\vspace{0.1cm}

%\begin{eqnarray*}

%\end{eqnarray*}

%\begin{propriedade}

%\end{propriedade}

%\begin{definicao}

%\end{definicao}

%\begin{ex}

%\end{ex}

%\begin{solucao}

%\end{solucao}

%\begin{teorema}

%\end{teorema}

%\begin{proof}

%\end{proof}

%\begin{corolario}

%\end{corolario}

%\begin{proposicao}

%\end{proposicao}

%\begin{flushright}
%\begin{minipage}{7cm}
%\small
%\end{minipage}\vspace{1cm}
%\end{flushright}